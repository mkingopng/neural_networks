%! Author = noone
%! Date = 10/9/22

% Preamble
\documentclass[11pt]{article}

% Packages
\usepackage{amsmath}

% Document
\begin{document}


CHAPTER 1. INTRODUCTION

Inventors have long dreamed of creating machines that think.
This desire dates back to at least the time of ancient Greece.
The mythical figures Pygmalion, Daedalus, and Hephaestus may all be interpreted as legendary inventors, and Galatea, Talos, and Pandora may all be regarded as artificial life (Ovid and Martin, 2004; Sparkes, 1996; Tandy, 1997).

When programmable computers were first conceived, people wondered whether such machines might become intelligent, over a hundred years before one was built (Lovelace, 1842).
Today, artificial intelligence (AI) is a thriving field with many practical applications and active research topics.
We look to intelligent software to automate routine labor, understand speech or images, make diagnoses in medicine and support basic scientific research.

In the early days of artificial intelligence, the field rapidly tackled and solved problems that are intellectually difficult for human beings but relatively straight-forward for computers—problems that can be described by a list of formal, mathematical rules.
The true challenge to artificial intelligence proved to be solving the tasks that are easy for people to perform but hard for people to describe formally—problems that we solve intuitively, that feel automatic, like recognizing spoken words or faces in images.

This book is about a solution to these more intuitive problems.
This solution is to allow computers to learn from experience and understand the world in terms of a hierarchy of concepts, with each concept defined through its relation to simpler concepts.
By gathering knowledge from experience, this approach avoids the need for human operators to formally specify all the knowledge that the computer needs.
The hierarchy of concepts enables the computer to learn complicated concepts by building them out of simpler ones.
If we draw a graph showing how these concepts are built on top of each other, the graph is deep, with many layers.
For this reason, we call this approach to AI deep learning.

Many of the early successes of AI took place in relatively sterile and formal environments and did not require computers to have much knowledge about the world.
For example, IBM’s Deep Blue chess-playing system defeated world champion Garry Kasparov in 1997 (Hsu, 2002).
Chess is of course a very simple world, containing only sixty-four locations and thirty-two pieces that can move in only rigidly circumscribed ways.
Devising a successful chess strategy is a tremendous accomplishment, but the challenge is not due to the difficulty of describing the set of chess pieces and allowable moves to the computer.
Chess can be completely described by a very brief list of completely formal rules, easily provided ahead of time by the programmer.

Ironically, abstract and formal tasks that are among the most difficult mental undertakings for a human being are among the easiest for a computer.
Computers have long been able to defeat even the best human chess player but only recently have begun matching some abilities of average human beings to recognize objects or speech.
A person’s everyday life requires an immense amount of knowledge about the world.
Much of this knowledge is subjective and intuitive, and therefore difficult to articulate in a formal way.
Computers need to capture this same knowledge in order to behave in an intelligent way.
One of the key challenges in artificial intelligence is how to get this informal knowledge into a computer.

Several artificial intelligence projects have sought to hard-code knowledge about the world in formal languages.
A computer can reason automatically about statements in these formal languages using logical inference rules.
This is known as the knowledge base approach to artificial intelligence.
None of these projects has led to a major success.
One of the most famous such projects is Cyc (Lenat and Guha, 1989).
Cyc is an inference engine and a database of statements in a language called CycL.
These statements are entered by a staff of human supervisors.
It is an unwieldy process.
People struggle to devise formal rules with enough complexity to accurately describe the world.
For example, Cyc failed to understand a story about a person named Fred shaving in the morning (Linde, 1992).
Its inference engine detected an inconsistency in the story: it knew that people do not have electrical parts, but because Fred was holding an electric razor, it believed the entity “FredWhileShaving” contained electrical parts.
It therefore asked whether Fred was still a person while he was shaving.

The difficulties faced by systems relying on hard-coded knowledge suggest that AI systems need the ability to acquire their own knowledge, by extracting patterns from raw data.
This capability is known as machine learning.
The introduction of machine learning enabled computers to tackle problems involving knowledge of the real world and make decisions that appear subjective.
A simple machine learning algorithm called logistic regression can determine whether to recommend cesarean delivery (Mor-Yosef et al., 1990).
A simple machine learning algorithm called naive Bayes can separate legitimate e-mail from spam e-mail.
The performance of these simple machine learning algorithms depends heavily on the representation of the data they are given.
For example, when logistic regression is used to recommend cesarean delivery, the AI system does not examine the patient directly.
Instead, the doctor tells the system several pieces of relevant information, such as the presence or absence of a uterine scar.
Each piece of information included in the representation of the patient is known as a feature.
Logistic regression learns how each of these features of the patient correlates with various outcomes.
However, it cannot influence how features are defined in any way.
If logistic regression were given an MRI scan of the patient, rather than the doctor’s formalized report, it would not be able to make useful predictions.
Individual pixels in an MRI scan have negligible correlation with any complications that might occur during delivery.
This dependence on representations is a general phenomenon that appears throughout computer science and even daily life.
In computer science, operations such as searching a collection of data can proceed exponentially faster if the collection is structured and indexed intelligently.
People can easily perform arithmetic on Arabic numerals but find arithmetic on Roman numerals much more time consuming.
It is not surprising that the choice of representation has an enormous effect on the performance of machine learning algorithms.
For a simple visual example, see figure 1.1.
Many artificial intelligence tasks can be solved by designing the right set of features to extract for that task, then providing these features to a simple machine learning algorithm.
For example, a useful feature for speaker identification from sound is an estimate of the size of the speaker’s vocal tract.
This feature gives a strong clue whether the speaker is a man, woman, or child.
For many tasks, however, it is difficult to know what features should be extracted.
For example, suppose that we would like to write a program to detect cars in photographs.
We know that cars have wheels, so we might like to use the presence of a wheel as a feature.
Unfortunately, it is difficult to describe exactly what a wheel looks like in terms of pixel values.
A wheel has a simple geometric shape, but its image may be complicated by shadows falling on the wheel, the sun glaring off the metal parts of the wheel, the fender of the car or an object in the foreground obscuring part of the wheel, and so on.


Figure 1.1:
Example of different representations: suppose we want to separate two
categories of data by drawing a line between them in a scatterplot.
In the plot on the left, we represent some data using Cartesian coordinates, and the task is impossible.
In the plot on the right, we represent the data with polar coordinates and the task becomes simple to
solve with a vertical line. (Figure produced in collaboration with David Warde-Farley.)


One solution to this problem is to use machine learning to discover not only the mapping from representation to output but also the representation itself.
This approach is known as representation learning.
Learned representations often result in much better performance than can be obtained with hand-designed representations.
They also enable AI systems to rapidly adapt to new tasks, with minimal human intervention.
A representation learning algorithm can discover a good set of features for a simple task in minutes, or for a complex task in hours to months.
Manually designing features for a complex task requires a great deal of human time and effort;
it can take decades for an entire community of researchers.

The quintessential example of a representation learning algorithm is the autoencoder.
An autoencoder is the combination of an encoder function, which converts the input data into a different representation, and a decoder function, which converts the new representation back into the original format.
Autoencoders are trained to preserve as much information as possible when an input is run through the encoder and then the decoder, but they are also trained to make the new representation have various nice properties.
Different kinds of autoencoders aim to achieve different kinds of properties.

When designing features or algorithms for learning features, our goal is usually to separate the factors of variation that explain the observed data.
In this context, we use the word “factors” simply to refer to separate sources of influence;
the factors are usually not combined by multiplication.
Such factors are often not quantities that are directly observed.
Instead, they may exist as either unobserved objects or unobserved forces in the physical world that affect observable quantities.
They may also exist as constructs in the human mind that provide useful simplifying explanations or inferred causes of the observed data.
They can be thought of as concepts or abstractions that help us make sense of the rich variability in the data.
When analyzing a speech recording, the factors of variation include the speaker’s
age, their sex, their accent and the words they are speaking.
When analyzing an image of a car, the factors of variation include the position of the car, its color, and the angle and brightness of the sun.

A major source of difficulty in many real-world artificial intelligence applications is that many of the factors of variation influence every single piece of data we are able to observe.
The individual pixels in an image of a red car might be very close to black at night.
The shape of the car’s silhouette depends on the viewing angle.
Most applications require us to disentangle the factors of variation and discard the ones that we do not care about.
Of course, it can be very difficult to extract such high-level, abstract features from raw data.
Many of these factors of variation, such as a speaker’s accent, can be identified only using sophisticated, nearly human-level understanding of the data.
When it is nearly as difficult to obtain a representation as to solve the original problem, representation learning does not, at first glance, seem to help us.

Deep learning solves this central problem in representation learning by introducing representations that are expressed in terms of other, simpler representations.
Deep learning enables the computer to build complex concepts out of simpler concepts.
Figure 1.2 shows how a deep learning system can represent the concept of an image of a person by combining simpler concepts, such as corners and contours, which are in turn defined in terms of edges.
The quintessential example of a deep learning model is the feedforward deep network, or multilayer perceptron (MLP).
A multilayer perceptron is just a mathematical function mapping some set of input values to output values.
The function is formed by composing many simpler functions.
We can think of each application of a different mathematical function as providing a new representation of the input.

The idea of learning the right representation for the data provides one perspective on deep learning.
Another perspective on deep learning is that depth enables the computer to learn a multistep computer program.
Each layer of the representation can be thought of as the state of the computer’s memory after executing another set of instructions in parallel.
Networks with greater depth can execute more instructions in sequence.
Sequential instructions offer great power because later instructions can refer back to the results of earlier instructions.
According to this view of deep learning, not all the information in a layer’s activations necessarily encodes factors of variation that explain the input.
The representation also stores state information that helps to execute a program that can make sense of the input.
This state information could be analogous to a counter or pointer in a traditional computer program.
It has nothing to do with the content of the input specifically, but it helps the model to organize its processing.

There are two main ways of measuring the depth of a model.
The first view is based on the number of sequential instructions that must be executed to evaluate the architecture.
We can think of this as the length of the longest path through a flowchart that describes how to compute each of the model’s outputs given its inputs.
Just as two equivalent computer programs will have different lengths depending on which language the program is written in, the same function may be drawn as a flowchart with different depths depending on which functions we allow to be used as individual steps in the flowchart.
Figure 1.3 illustrates how this choice of language can give two different measurements for the same architecture.

Another approach, used by deep probabilistic models, regards the depth of a model as being not the depth of the computational graph but the depth of the graph describing how concepts are related to each other.
In this case, the depth of the flowchart of the computations needed to compute the representation of each concept may be much deeper than the graph of the concepts themselves.
This is because the system’s understanding of the simpler concepts can be refined given information about the more complex concepts.
For example, an AI system observing an image of a face with one eye in shadow may initially see only one eye.
After detecting that a face is present, the system can then infer that a second eye is probably present as well.
In this case, the graph of concepts includes only two layers—a layer for eyes and a layer for faces—but the graph of computations includes 2n layers if we refine our estimate of each concept given the other n times.

Because it is not always clear which of these two views—the depth of the
computational graph, or the depth of the probabilistic modeling graph—is most
relevant, and because different people choose different sets of smallest elements
from which to construct their graphs, there is no single correct value for the
depth of an architecture, just as there is no single correct value for the length of
a computer program. Nor is there a consensus about how much depth a model
requires to qualify as “deep.” However, deep learning can be safely regarded as the
study of models that involve a greater amount of composition of either learned
functions or learned concepts than traditional machine learning does.

To summarize, deep learning, the subject of this book, is an approach to AI.
Specifically, it is a type of machine learning, a technique that enables computer
systems to improve with experience and data. We contend that machine learning
is the only viable approach to building AI systems that can operate in complicated
real-world environments. Deep learning is a particular kind of machine learning
that achieves great power and flexibility by representing the world as a nested
hierarchy of concepts, with each concept defined in relation to simpler concepts, and
more abstract representations computed in terms of less abstract ones. Figure 1.4
illustrates the relationship between these different AI disciplines. Figure 1.5 gives
a high-level schematic of how each works.

1.1 Who should read this book?

This book can be useful for a variety of readers, but we wrote it with two target audiences in mind.
One of these target audiences is university students (undergraduate or graduate) learning about machine learning, including those who are beginning a career in deep learning and artificial intelligence research.
The other target audience is software engineers who do not have a machine learning or statistics background but want to rapidly acquire one and begin using deep learning in their product or platform.
Deep learning has already proved useful in many software disciplines, including computer vision, speech and audio processing, natural language processing, robotics, bioinformatics and chemistry, video games, search engines, online advertising and finance.

This book has been organized into three parts to best accommodate a variety of readers.
Part I introduces basic mathematical tools and machine learning concepts.
Part II describes the most established deep learning algorithms, which are essentially solved technologies.
Part III describes more speculative ideas that are widely believed to be important for future research in deep learning.

figure 1-5

Readers should feel free to skip parts that are not relevant given their interests or background.
Readers familiar with linear algebra, probability, and fundamental machine learning concepts can skip part I, for example, while those who just want to implement a working system need not read beyond part II.
To help choose which chapters to read, figure 1.6 provides a flowchart showing the high-level organization of the book.

fig 1-6

We do assume that all readers come from a computer science background.
We assume familiarity with programming, a basic understanding of computational performance issues, complexity theory, introductory level calculus and some of the terminology of graph theory.

1.2 Historical Trends in Deep Learning

It is easiest to understand deep learning with some historical context.
Rather than providing a detailed history of deep learning, we identify a few key trends:
• Deep learning has had a long and rich history, but has gone by many names, reflecting different philosophical viewpoints, and has waxed and waned in popularity.
• Deep learning has become more useful as the amount of available training data has increased.
• Deep learning models have grown in size over time as computer infrastructure (both hardware and software) for deep learning has improved.
• Deep learning has solved increasingly complicated applications with increasing accuracy over time.

1.2.1 The many Names and Changing Fortunes of Neural Networks

We expect that many readers of this book have heard of deep learning as an exciting new technology, and are surprised to see a mention of “history” in a book about an emerging field.
In fact, deep learning dates back to the 1940s.
Deep learning only appears to be new, because it was relatively unpopular for several years preceding its current popularity, and because it has gone through many different names, only recently being called “deep learning.”
The field has been rebranded many times, reflecting the influence of different researchers and different perspectives.

A comprehensive history of deep learning is beyond the scope of this textbook.
Some basic context, however, is useful for understanding deep learning.
Broadly speaking, there have been three waves of development:
- deep learning known as cybernetics in the 1940s–1960s,
- deep learning known as connectionism in the 1980s–1990s,
- and the current resurgence under the name deep learning beginning in 2006.
This is quantitatively illustrated in figure 1.7.

fig 1-7

Some of the earliest learning algorithms we recognize today were intended to be computational models of biological learning, that is, models of how learning happens or could happen in the brain.
As a result, one of the names that deep learning has gone by is artificial neural networks(ANNs).
The corresponding perspective on deep learning models is that they are engineered systems inspired by the biological brain (whether the human brain or the brain of another animal).
While the kinds of neural networks used for machine learning have sometimes been used to understand brain function (Hinton and Shallice, 1991), they are generally not designed to be realistic models of biological function.
The neural perspective on deep learning is motivated by two main ideas.
One idea is that the brain provides a proof by example that intelligent behavior is possible, and a conceptually straightforward path to building intelligence is to reverse engineer the computational principles behind the brain and duplicate its functionality.
Another perspective is that it would be deeply interesting to understand the brain and the principles that underlie human intelligence, so machine learning models that shed light on these basic scientific questions are useful apart from their ability to solve engineering applications.
The modern term “deep learning” goes beyond the neuroscientific perspective on the current breed of machine learning models.
It appeals to a more general principle of learning multiple levels of composition, which can be applied in machine learning frameworks that are not necessarily neurally inspired.
The earliest predecessors of modern deep learning were simple linear models motivated from a neuroscientific perspective.
These models were designed to take a set of $n$ input values $x_1,..., X_n$ and associate them with an output $y$.
These models would learn a set of weights $w_1,...w_n$ and compute their output $f(x,w) = x_1 w_1 + ... + x_n w_n$.
This first wave of neural networks research was known as cybernetics, as illustrated in figure 1.7.

The McCulloch-Pitts neuron (McCulloch and Pitts, 1943) was an early model of brain function.
This linear model could recognize two different categories of inputs by testing whether $f(x,w)$ is positive or negative.
Of course, for the model to correspond to the desired definition of the categories, the weights needed to be set correctly.
These weights could be set by the human operator.
In the 1950s, the perceptron (Rosenblatt, 1958, 1962) became the first model that could learn the weights that defined the categories given examples of inputs from each category.
The adaptive linear element(ADALINE), which dates from about the same time, simply returned the value of $f(x)$ itself to predict a real number (Widrow and Hoff, 1960) and could also learn to predict these numbers from data.

These simple learning algorithms greatly affected the modern landscape of machine learning.
The training algorithm used to adapt the weights of the ADALINE was a special case of an algorithm called stochastic gradient descent.
Slightly modified versions of the stochastic gradient descent algorithm remain the dominant training algorithms for deep learning models today.
Models based on the used by the perceptron and ADALINE are called linear models.
These models remain some of the most widely used machine learning models, though in many cases they are trained in different ways than the original models were trained.

Linear models have many limitations.
Most famously, they cannot learn the XOR function, where $f([0,1],w) = 1$ but $f([1,1],w)=0$ and $f([0,0],w=0)$.
Critics who observed these flaws in linear models caused a backlash against biologically inspired learning in general (Minsky and Papert, 1969).
This was the first major dip in the popularity of neural networks.

Today, neuroscience is regarded as an important source of inspiration for deep learning researchers, but it is no longer the predominant guide for the field.
The main reason for the diminished role of neuroscience in deep learning research today is that we simply do not have enough information about the brain to use it as a guide.
To obtain a deep understanding of the actual algorithms used by the brain, we would need to be able to monitor the activity of (at the very least) thousands of interconnected neurons simultaneously.
Because we are not able to do this, we are far from understanding even some of the most simple and well-studied parts of the brain (Olshausen and Field, 2005).
Neuroscience has given us a reason to hope that a single deep learning algorithm can solve many different tasks.
Neuroscientists have found that ferrets can learn to “see” with the auditory processing region of their brain if their brains are rewired to send visual signals to that area (Von Melchner et al., 2000).
This suggests that much of the mammalian brain might use a single algorithm to solve most of the different tasks that the brain solves.
Before this hypothesis, machine learning research was more fragmented, with different communities of researchers studying natural language processing, vision, motion planning and speech recognition.
Today,these application communities are still separate, but it is common for deep learning research groups to study many or even all these application areas simultaneously.

We are able to draw some rough guidelines from neuroscience.
The basic idea of having many computational units that become intelligent only via their interactions with each other is inspired by the brain.
The neocognitron (Fukushima,1980) introduced a powerful model architecture for processing images that was inspired by the structure of the mammalian visual system and later became the basis for the modern convolutional network (LeCun et al., 1998b), as we will see in section 9.10.
Most neural networks today are based on a model neuron called the rectified linear unit.
The original cognitron (Fukushima, 1975) introduced a more complicated version that was highly inspired by our knowledge of brain function.
The simplified modern version was developed incorporating ideas from many viewpoints, with Nair and Hinton (2010) and Glorot et al. (2011a) citing neuroscience as an influence, and Jarrett et al. (2009) citing more engineering-oriented influences.
While neuroscience is an important source of inspiration, it need not be taken as a rigid guide.
We know that actual neurons compute very different functions than modern rectified linear units, but greater neural realism has not yet led to an improvement in machine learning performance.
Also, while neuroscience has successfully inspired several neural network architectures, we do not yet know enough about biological learning for neuroscience to offer much guidance for the learning algorithms we use to train these architectures.

Media accounts often emphasize the similarity of deep learning to the brain.
While it is true that deep learning researchers are more likely to cite the brain as an influence than researchers working in other machine learning fields, such as kernel machines or Bayesian statistics, one should not view deep learning as an attempt to simulate the brain.
Modern deep learning draws inspiration from many fields, especially applied math fundamentals like linear algebra, probability, information theory, and numerical optimization.
While some deep learning researchers cite neuroscience as an important source of inspiration, others are not concerned with neuroscience at all.
It is worth noting that the effort to understand how the brain works on an algorithmic level is alive and well.
This endeavor is primarily known as “computational neuroscience” and is a separate field of study from deep learning.
It is common for researchers to move back and forth between both fields.
The field of deep learning is primarily concerned with how to build computer systems that are able to successfully solve tasks requiring intelligence, while the field of computational neuroscience is primarily concerned with building more accurate models of how the brain actually works.
In the 1980s, the second wave of neural network research emerged in great part via a movement called connectionism, or parallel distributed processing (Rumelhart et al., 1986c; McClelland et al., 1995).
Connectionism arose in the context of cognitive science.
Cognitive science is an interdisciplinary approach to understanding the mind, combining multiple different levels of analysis.
During the early 1980s, most cognitive scientists studied models of symbolic reasoning.
Despite their popularity, symbolic models were difficult to explain in terms of how the brain could actually implement them using neurons.
The connectionists began to study models of cognition that could actually be grounded in neural implementations (Touretzky and Minton, 1985), reviving many ideas dating back to the work of psychologist Donald Hebb in the 1940s (Hebb, 1949).
The central idea in connectionism is that a large number of simple computational units can achieve intelligent behavior when networked together.
This insight applies equally to neurons in biological nervous systems as it does to hidden units in computational models.
Several key concepts arose during the connectionism movement of the 1980s that remain central to today’s deep learning.

One of these concepts is that of distributed representation (Hinton et al., 1986).
This is the idea that each input to a system should be represented by many features, and each feature should be involved in the representation of many possible inputs.
For example, suppose we have a vision system that can recognize cars, trucks, and birds, and these objects can each be red, green, or blue.
One way of representing these inputs would be to have a separate neuron or hidden unit that activates for each of the nine possible combinations: red truck, red car, red bird, green truck, and so on.
This requires nine different neurons, and each neuron must independently learn the concept of color and object identity.
One way to improve on this situation is to use a distributed representation, with three neurons describing the color and three neurons describing the object identity.
This requires only six neurons total instead of nine, and the neuron describing redness is able to learn about redness from images of cars, trucks and birds, not just from images of one specific category of objects.
The concept of distributed representation is central to this book and is described in greater detail in chapter 15.

Another major accomplishment of the connectionist movement was the successful use of back-propagation to train deep neural networks with internal representations and the popularization of the back-propagation algorithm (Rumelhart et al., 1986a; LeCun, 1987).
This algorithm has waxed and waned in popularity but, as of this writing, is the dominant approach to training deep models.
During the 1990s, researchers made important advances in modeling sequences with neural networks.
Hochreiter (1991) and Bengio et al. (1994) identified some of the fundamental mathematical difficulties in modeling long sequences, described in section 10.7.
Hochreiter and Schmidhuber (1997) introduced the long short-term memory (LSTM) network to resolve some of these difficulties.
Today, the LSTM is widely used for many sequence modeling tasks, including many natural language processing tasks at Google.
The second wave of neural networks research lasted until the mid-1990s.
Ventures based on neural networks and other AI technologies began to make unrealistically ambitious claims while seeking investments.
When AI research did not fulfill these unreasonable expectations, investors were disappointed.
Simultaneously, other fields of machine learning made advances.
Kernel machines (Boser et al.,1992; Cortes and Vapnik, 1995; Schölkopf et al., 1999) and graphical models (Jor-dan, 1998) both achieved good results on many important tasks.
These two factors led to a decline in the popularity of neural networks that lasted until 2007.

During this time, neural networks continued to obtain impressive performance on some tasks (LeCun et al., 1998b; Bengio et al., 2001).
The Canadian Institute for Advanced Research (CIFAR) helped to keep neural networks research alive via its Neural Computation and Adaptive Perception (NCAP) research initiative.
This program united machine learning research groups led by Geoffrey Hinton at University of Toronto, Yoshua Bengio at University of Montreal, and Yann LeCun at New York University.
The multidisciplinary CIFAR NCAP research initiative also included neuroscientists and experts in human and computer vision.

At this point, deep networks were generally believed to be very difficult to train.
We now know that algorithms that have existed since the 1980s work quite well, but this was not apparent circa 2006.
The issue is perhaps simply that these algorithms were too computationally costly to allow much experimentation with the hardware available at the time.

The third wave of neural networks research began with a breakthrough in 2006.
Geoffrey Hinton showed that a kind of neural network called a deep belief network could be efficiently trained using a strategy called greedy layer-wise pretraining (Hinton et al., 2006), which we describe in more detail in section 15.1.
The other CIFAR-affiliated research groups quickly showed that the same strategy could be used to train many other kinds of deep networks (Bengio et al., 2007; Ranzato et al., 2007a) and systematically helped to improve generalization on test examples.
This wave of neural networks research popularized the use of the term “deep learning” to emphasize that researchers were now able to train deeper neural networks than had been possible before, and to focus attention on the theoretical importance of depth (Bengio and LeCun, 2007; Delalleau and Bengio, 2011; Pascanu et al., 2014a; Montufar et al., 2014).
At this time, deep neural networks outperformed competing AI systems based on other machine learning technologies as well as hand-designed functionality.
This third wave of popularity of neural networks continues to the time of this writing, though the focus of deep learning research has changed dramatically within the time of this wave.
The third wave began with a focus on new unsupervised learning techniques and the ability of deep models to generalize well from small datasets, but today there is more interest in much older supervised learning algorithms and the ability of deep models to leverage large labeled datasets.

1.2.2 Increasing Dataset Sizes

One may wonder why deep learning has only recently become recognized as a crucial technology even though the first experiments with artificial neural networks were conducted in the 1950s.
Deep learning has been successfully used in commercial applications since the 1990s but was often regarded as being more of an art than a technology and something that only an expert could use, until recently.
It is true that some skill is required to get good performance from a deep learning algorithm.
Fortunately, the amount of skill required reduces as the amount of training data increases.
The learning algorithms reaching human performance on complex tasks today are nearly identical to the learning algorithms that struggled to solve toy problems in the 1980s, though the models we train with these algorithms have undergone changes that simplify the training of very deep architectures.
The most important new development is that today we can provide these algorithms with the resources they need to succeed.
Figure 1.8 shows how the size of benchmark datasets has expanded remarkably over time.
This trend is driven by the increasing digitization of society.
As more and more of our activities take place on computers,more and more of what we do is recorded.
As our computers are increasingly networked together, it becomes easier to centralize these records and curate them into a dataset appropriate for machine learning applications.
The age of “Big Data” has made machine learning much easier because the key burden of statistical estimation—generalizing well to new data after observing only a small amount of data—has been considerably lightened.
As of 2016, a rough rule of thumb is that a supervised deep learning algorithm will generally achieve acceptable performance with around 5,000 labeled examples per category and will match or exceed human performance when trained with a dataset containing at least 10 million labeled examples.
Working successfully with datasets smaller than this is an important research area, focusing in particular on how we can take advantage of large quantities of unlabeled examples, with unsupervised or semi-supervised learning.

fig 1-8

fig 1-9

1.2.3 Increasing Model Sizes

Another key reason that neural networks are wildly successful today after enjoying comparatively little success since the 1980s is that we have the computational resources to run much larger models today.
One of the main insights of connectionism is that animals become intelligent when many of their neurons work together.
An individual neuron or small collection of neurons is not particularly useful.
Biological neurons are not especially densely connected.
As seen in figure 1.10, our machine learning models have had a number of connections per neuron within an order of magnitude of even mammalian brains for decades.
In terms of the total number of neurons, neural networks have been astonishingly small until quite recently, as shown in figure 1.11.
Since the introduction of hidden units, artificial neural networks have doubled in size roughly every 2.4 years.
This growth is driven by faster computers with larger memory and by the availability of larger datasets.
Larger networks are able to achieve higher accuracy on more complex tasks.
This trend looks set to continue for decades.
Unless new technologies enable faster scaling, artificial neural networks will not have the same number
of neurons as the human brain until at least the 2050s.
Biological neurons may represent more complicated functions than current artificial neurons, so biological neural networks may be even larger than this plot portrays.
In retrospect, it is not particularly surprising that neural networks with fewer neurons than a leech were unable to solve sophisticated artificial intelligence problems.
Even today’s networks, which we consider quite large from a computational systems point of view, are smaller than the nervous system of even relatively primitive vertebrate animals like frogs.
The increase in model size over time, due to the availability of faster CPUs, the advent of general purpose GPUs (described in section 12.1.2), faster network connectivity and better software infrastructure for distributed computing, is one of the most important trends in the history of deep learning.
This trend is generally expected to continue well into the future.

fig 1-10
1.2.4 Increasing Accuracy, Complexity and Real-World Impact

Since the 1980s, deep learning has consistently improved in its ability to provide accurate recognition and prediction.
Moreover, deep learning has consistently been applied with success to broader and broader sets of applications.
The earliest deep models were used to recognize individual objects in tightly cropped, extremely small images (Rumelhart et al., 1986a).
Since then there has been a gradual increase in the size of images neural networks could process.
Modern object recognition networks process rich high-resolution photographs and do not have a requirement that the photo be cropped near the object to be recognized (Krizhevsky et al., 2012).
Similarly, the earliest networks could recognize only two kinds of objects (or in some cases, the absence or presence of a single kind of object), while these modern networks typically recognize at least 1,000 different categories of objects.
The largest contest in object recognition is the ImageNet Large Scale Visual Recognition Challenge (ILSVRC) held each year. A dramatic
moment in the meteoric rise of deep learning came when a convolutional network
won this challenge for the first time and by a wide margin, bringing down the
state-of-the-art top-5 error rate from 26.1 percent to 15.3 percent (Krizhevsky
et al., 2012), meaning that the convolutional network produces a ranked list of
possible categories for each image, and the correct category appeared in the first
five entries of this list for all but 15.3 percent of the test examples. Since then,
these competitions are consistently won by deep convolutional nets, and as of this
writing, advances in deep learning have brought the latest top-5 error rate in this
contest down to 3.6 percent, as shown in figure 1.12.

Deep learning has also had a dramatic impact on speech recognition.
After improving throughout the 1990s, the error rates for speech recognition stagnated starting in about 2000.
The introduction of deep learning (Dahl et al., 2010; Deng et al., 2010b; Seide et al., 2011; Hinton et al., 2012a) to speech recognition resulted in a sudden drop in error rates, with some error rates cut in half.
We explore this history in more detail in section 12.3.
Deep networks have also had spectacular successes for pedestrian detection and image segmentation (Sermanet et al., 2013; Farabet et al., 2013; Couprie et al.,2013) and yielded superhuman performance in traffic sign classification (Ciresan et al., 2012).
At the same time that the scale and accuracy of deep networks have increased, so has the complexity of the tasks that they can solve.
Goodfellow et al. (2014d) showed that neural networks could learn to output an entire sequence of characters transcribed from an image, rather than just identifying a single object.
Previously, it was widely believed that this kind of learning required labeling of the individual elements of the sequence (Gülçehre and Bengio, 2013).
Recurrent neural networks, such as the LSTM sequence model mentioned above, are now used to model relationships between sequences and other sequences rather than just fixed inputs.
This sequence-to-sequence learning seems to be on the cusp of revolutionizing another application: machine translation (Sutskever et al., 2014; Bahdanau et al., 2015).
This trend of increasing complexity has been pushed to its logical conclusion with the introduction of neural Turing machines (Graves et al., 2014) that learn to read from memory cells and write arbitrary content to memory cells.
Such neural networks can learn simple programs from examples of desired behavior.
For example, they can learn to sort lists of numbers given examples of scrambled and sorted sequences.
This self-programming technology is in its infancy, but in the future it could in principle be applied to nearly any task.
Another crowning achievement of deep learning is its extension to the domain of reinforcement learning.
In the context of reinforcement learning, an autonomous agent must learn to perform a task by trial and error, without any guidance from the human operator.
DeepMind demonstrated that a reinforcement learning system based on deep learning is capable of learning to play Atari video games, reaching human-level performance on many tasks (Mnih et al., 2015).
Deep learning has also significantly improved the performance of reinforcement learning for robotics (Finn et al., 2015).
Many of these applications of deep learning are highly profitable.
Deep learning is now used by many top technology companies, including Google, Microsoft, Facebook, IBM, Baidu, Apple, Adobe, Netflix, NVIDIA, and NEC.

Advances in deep learning have also depended heavily on advances in software infrastructure.
Software libraries such as Theano (Bergstra et al., 2010; Bastien et al., 2012), PyLearn2 (Goodfellow et al., 2013c), Torch (Collobert et al., 2011b), DistBelief (Dean et al., 2012), Caffe (Jia, 2013), MXNet (Chen et al., 2015), and TensorFlow (Abadi et al., 2015) have all supported important research projects or commercial products.
Deep learning has also made contributions to other sciences.
Modern convolutional networks for object recognition provide a model of visual processing that neuroscientists can study (DiCarlo, 2013).
Deep learning also provides useful tools for processing massive amounts of data and making useful predictions in scientific fields.
It has been successfully used to predict how molecules will interact in order to help pharmaceutical companies design new drugs (Dahl et al., 2014), to search for subatomic particles (Baldi et al., 2014), and to automatically parse microscope images used to construct a 3-D map of the human brain (Knowles-Barley et al., 2014).
We expect deep learning to appear in more and more scientific fields in the future.

In summary, deep learning is an approach to machine learning that has drawn heavily on our knowledge of the human brain, statistics and applied math as it developed over the past several decades.
In recent years, deep learning has seen tremendous growth in its popularity and usefulness, largely as the result of more powerful computers, larger datasets and techniques to train deeper networks.
The years ahead are full of challenges and opportunities to improve deep learning even further and to bring it to new frontiers.

\end{document}